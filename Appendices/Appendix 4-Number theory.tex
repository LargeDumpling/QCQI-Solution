\documentclass{article}
\usepackage{amsmath}
\usepackage{amssymb}
\usepackage{amsthm}
\usepackage{geometry}
\geometry{a4paper,scale=0.85}
\title{Appendix 4: Number theory}
\author{Hongyuan Sun\thanks{E-mail:LargeDumpling@foxmail.com}}
\begin{document}
\maketitle
\section*{A4.1 Fundamentals}

\section*{A4.2 Modular arithmetic and Euclid's algorithm}

\section*{A4.3 Reduction of factoriing to order-finding}

\section*{A4.4 Continued fraction}
	\subsection*{Exercise A4.18}
	\begin{align*}
		x &= \frac{19}{17} = [1,8,2] \tag{AE4.18-1} \\
		x &= \frac{77}{65} = [1,5,2,2,2] \tag{AE4.18-2}
	\end{align*}
	
	\subsection*{Exercise A4.19}
	\begin{proof}
		\[ \because \begin{cases}
			[a_0] = a_0 = \frac{p_0}{q_0} \\
			[a_0,a_1] = a_0 + \frac{1}{a_1} = \frac{p_1}{q_1} \tag{AE4.19-1}
		\end{cases} \]
		\[ \therefore \begin{cases}
			p_0 = a_0 \\
			q_0 = 1
		\end{cases} \begin{cases}
			p_1 = a_0a_1 + 1 \\
			q_1 = a_1 \tag{AE4.19-2}
		\end{cases} \]
		\[ \therefore q_1p_0 - p_1q_0 = a_0a_1 - (a_0a_1+1) = -1 \]

		If $q_np_{n-1} - p_nq_{n-1} = (-1)^n$ for $n=k$, then with equations (AE4.42),(AE4.43), we can get
		\begin{align*}
			q_{k+1}p_k - p_{k+1}q_k
			&= (a_{k+1}q_k + q_{k-1})p_k - (a_{k+1}p_k + p_{k-1})q_k \tag{AE4.19-3} \\
			&= a_{k+1}(q_kp_k - p_kq_k)+q_{k-1}p_k - p_{k-1}q_k \tag{AE4.19-4} \\
			&= - (-1)^k = (-1)^{k+1} \tag{AE4.19-5}
		\end{align*}

		So $q_np_{n-1} - p_nq_{n-1} = (-1)^n$ for $n=k+1$, using inductive reasoning we can prove that this statement is true for $n \geqslant 1$.
	\end{proof}

	\subsection*{Problem 4.1(Prime number estimate)}
		\subsubsection*{(1)}
		\begin{proof}
			\begin{align*}
				\log \binom{2n}{n}
				&= \log \frac{\prod_{i=0}^{n-1}(2n - i)}{\prod_{i=0}^{n-1}(n - i)} \tag{AP4.1-1} \\
				&= \log \prod_{i=0}^{n-1} \frac{2n - i}{n - i} \tag{AP4.1-2} \\
				&\geqslant \log 2^n \tag{AP4.1-3} \\
				&= n \qedhere
			\end{align*}
		\end{proof}

		\subsubsection*{(2)}
		
		\subsubsection*{(3)}
		\begin{proof}
			\begin{align*}
				\because n
				&\leqslant \log \binom{2n}{n} \tag{AP4.1-4} \\
				&\leqslant \sum_{p \leqslant 2n} \left\lfloor \frac{\log(2n)}{\log p} \right\rfloor \log p \tag{AP4.1-5} \\
				&\leqslant \sum_{p \leqslant 2n} \log(2n) \tag{AP4.1-6} \\
				&= \pi (2n) \log (2n) \tag{AP4.1-7}
			\end{align*}
			\[ \therefore \pi(2n) \geqslant \frac{n}{\log(2n)} \qedhere \]
		\end{proof}

\end{document}

